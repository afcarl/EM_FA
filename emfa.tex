\documentclass{article}

\usepackage[top=2.5cm, bottom=2.5cm, left=3cm, right=3cm]{geometry}
\usepackage{amsmath,amssymb}

\DeclareMathOperator{\diag}{diag}
\DeclareMathOperator*{\E}{\rm E}
\DeclareMathOperator*{\var}{var}
\DeclareMathOperator*{\cov}{cov}
\DeclareMathOperator*{\argmax}{\rm argmax}
\DeclareMathOperator*{\argmin}{\rm argmin}
\DeclareMathOperator{\tr}{tr}

\newcommand{\identity}{\ensuremath{\mathbb{I}}}
\newcommand{\gaussian}{\ensuremath{\mathcal{N}}}
\newcommand{\trp}{{^\top}} % transpose
\newcommand{\inv}{^{-1}}
\newcommand{\oneHalf}{\frac{1}{2}}
\newcommand{\hadamard}{\circ}

\usepackage{forloop}
\newcommand{\defvec}[1]{\expandafter\newcommand\csname v#1\endcsname{{\mathbf{#1}}}}
\newcounter{ct}
\forLoop{1}{26}{ct}{
    \edef\letter{\alph{ct}}
    \expandafter\defvec\letter
}
\forLoop{1}{26}{ct}{
    \edef\letter{\Alph{ct}}
    \expandafter\defvec\letter
}

\AtBeginDocument{\setlength\abovedisplayskip{5pt}}
\AtBeginDocument{\setlength\belowdisplayskip{5pt}}

\title{EM for factor analysis}
\author{I. Memming Park}
\begin{document}
\maketitle
Factor Analysis (FA) is usually not fit by expectation-maximization (EM), but we do it anyways.
%This document describes Memming's struggle in implementing EM for FA; he presented this in the computational neuroscience journal club at the University of Texas at Austin on August 6th, 2014.

The generative model for FA is given by~\cite{Roweis1999},
\begin{align}\label{eq:model}
    \vx_i &\sim \gaussian(0, \vI)\nonumber\\
    \vy_i &\sim \gaussian(\vC\vx_i, \vR)\\
    \vR &= \diag(\sigma_1^2, \cdots, \sigma_p^2)\nonumber
\end{align}
where $\vx$ is a $q$-dimensional latent variable,
$\vy$ is a $p$-dimensional observation, and $q < p$.
$\vC$ is a $(p \times q)$ factor loadings matrix, and
$\vR$ is the covariance representing independent noise in each observed
dimension.

Note that $\vx$ and $\vy$ are jointly normal, in fact~\cite{Bishop2006},
\begin{align}
    \begin{bmatrix}
	\vx_i\\\vy_i
    \end{bmatrix}
    &=
    \gaussian\left(0, \begin{bmatrix}
	    \vI & \vC\trp\\
	    \vC & \vR + \vC\vC\trp
	\end{bmatrix}
    \right).
\end{align}
Also, the posterior over the latents is given by,
\begin{align}\label{eq:latent:posterior}
    \vx_i | \vy_i &\sim \gaussian\left(\mu, \Lambda\right),
\end{align}
where
\begin{align}
    \label{eq:estep}
    \Lambda &= \left(\vI + \vC\trp \vR\inv \vC\right)\inv,\nonumber\\
    \mu_i &= \Lambda \vC \vR\inv \vy_i.
\end{align}
Note that the covariance of the latent doesn't depend on the sample.
Our goal is to find the parameters $\theta = \{\vC, \vR\}$ that maximizes the
likelihood $p(\vy|\theta)$.

In the E-step, we minimize the KL divergence between $Q(x)$ and
$p(\vx|\vy,\theta)$, which can be achieved by setting $Q(x) =
p(\vx|\vy,\theta)$. Hence, \eqref{eq:estep} corresponds to the computation
required for the E-step.

In the M-step, we maximize the conditional expectation of the total data
log-likelihood with respect to $\theta$, i.e.,
\begin{align}\label{eq:mstep:objective}
    \theta_{\mathrm{new}} =
	\argmax_\theta
	\E_{\vx\sim Q}
	    \left[
		\sum_i \log
		    p(\vy_i, \vx_i|\theta)
	    \right]
\end{align}
To compute the expectation, it's convenient to define the following quantities~\cite{Rubin1982}:
\begin{align}
    \delta &= \Lambda \vC \vR\inv\\
    \Sigma_{yy} &= \frac{1}{N}\sum_i \vy_i \vy_i\trp\\
    \Sigma_{yx} &= \frac{1}{N}\E_{\vx \sim Q} \left[ \sum_i \vy_i \vx_i\trp \right]
	= \frac{1}{N}\sum_i \vy_i \mu_i\trp
	= \frac{1}{N}\sum_i \vy_i \vy_i\trp \delta\trp
	= \Sigma_{yy} \delta\trp\\
    \Sigma_{xx} &= \frac{1}{N}\E_{\vx \sim Q} \left[ \sum_i \vx_i \vx_i\trp \right]
	= \frac{1}{N} \sum_i 
	    \left(
		\cov_{\vx\sim Q}(\vx_i) + \mu_i\mu_i\trp
	    \right)
	= \Lambda + \delta\Sigma_{yy}\delta\trp
\end{align}
Note that the total data log-likelihood can be written as,
\begin{align}
    \sum_i \log p(\vy_i, \vx_i|\theta) 
    &=
    \sum_i \log p(\vy_i| \vx_i\theta) 
    +
    \sum_i \log p(\vx_i),
\end{align}
and the second term does not depend on the parameters $\theta$, hence
our objective in \eqref{eq:mstep:objective} can simply be written as,
\begin{align}
\theta_{\mathrm{new}} =
    \argmax_\theta
    \E_{\vx\sim Q}
	\left[
	    \sum_i \log
		p(\vy_i, \vx_i|\theta)
	\right] =
\frac{1}{N}\E_{\vx\sim Q}
    \left[
	\sum_i \log
	    p(\vy_i| \vx_i, \theta)
    \right]
\end{align}
where $N$ is the number of i.i.d. samples.
\begin{align*}
\frac{1}{N} & \sum_{i=1}^N \log\left(
	p(\vy_i| \vx_i, \theta)
    \right)
    \nonumber\\
    &=
    -\oneHalf \log |\vR|
    -\oneHalf 
    \frac{1}{N}
    \sum_i
	\left(\vC \vx_i - \vy_i\right)\trp
	\vR\inv
	\left(\vC \vx_i - \vy_i\right)
    \\
    &=
    -\oneHalf \log |\vR|
    -\oneHalf 
    \frac{1}{N}
    \sum_i
	\left(
	    \vx_i\trp\vC\trp \vR\inv \vC \vx_i
	    +\vy_i\trp \vR\inv \vy_i
	    -2\vy_i\trp \vR\inv \vC \vx_i
	    %-\vx_i\trp\vC\trp \vR\inv \vy_i
	\right)
    \\
    &=
    -\oneHalf \log |\vR|
    -\oneHalf 
	\left(
	    \tr\left[
		\vC\trp \vR\inv \vC 
		\frac{1}{N} \sum_i
		    \vx_i\vx_i\trp
	    \right]
	    +
	    \tr\left[
		\vR\inv 
		\frac{1}{N} \sum_i
		    \vy_i \vy_i\trp 
	    \right]
	    -
	    2\tr\left[
		\vR\inv \vC 
		\frac{1}{N} \sum_i
		    \vx_i \vy_i\trp 
	    \right]
	\right)
\end{align*}
Now, taking the expectation over $\vx \sim Q$,
\begin{align}
\frac{1}{N}
    \E_{\vx\sim Q}&\left[
\sum_{i=1}^N \log\left(
	p(\vy_i| \vx_i, \theta)
    \right)
\right]
    \nonumber\\
    &=
    -\oneHalf \log |\vR|
    -\oneHalf 
	\left(
	    \tr\left[
		\vC\trp \vR\inv \vC 
		\Sigma_{xx}
	    \right]
	    +
	    \tr\left[
		\vR\inv 
		\Sigma_{yy}
	    \right]
	    -
	    2\tr\left[
		\vR\inv \vC 
		\Sigma_{xy}
	    \right]
	\right),
\end{align}
where constant terms are omitted.
Keep in mind that $\Sigma_{xx}$ and $\Sigma_{xy}$'s dependence on parameters
are through $Q$, and hence fixed during the M-step update.
We find the stationary points by inspecting the partial derivatives with
respect to the parameters.
\begin{align}\label{eq:mstep:C}
&\frac{\partial}{\partial C}
= \vR\inv \hat{\vC} \Sigma_{xx} - \vR\inv \Sigma_{xy}\trp = 0
\\
& \implies \hat{\vC} = \Sigma_{yx}\Sigma_{xx}\inv
= \Sigma_{yy} \delta\trp (\Lambda + \delta\Sigma_{yy}\delta\trp)\inv
\end{align}
%
\begin{align}\label{eq:mstep:R}
\frac{\partial}{\partial R\inv}
&= \oneHalf \vR -\oneHalf \vC\Sigma_{xx}\vC\trp \hadamard \vI 
- \oneHalf\Sigma_{yy} \hadamard \vI + \Sigma_{yx}\vC\trp \hadamard \vI  = 0
\\
\implies \vR &= 
\left(
    \vC\Sigma_{xx}\vC\trp + \Sigma_{yy} - 2\Sigma_{yx}\vC\trp
\right) \hadamard \vI
\\
&=
\left(
    \Sigma_{yy} \delta\trp (\Lambda + \delta\Sigma_{yy}\delta\trp)\inv
    %(\Lambda + \delta\Sigma_{yy}\delta\trp)
    %(\Lambda + \delta\Sigma_{yy}\delta\trp)\inv
    \delta
    \Sigma_{yy} 
    + \Sigma_{yy} 
    - 2\Sigma_{yy} \delta\trp
    (\Lambda + \delta\Sigma_{yy}\delta\trp)\inv
    \delta
    \Sigma_{yy} 
\right) \hadamard \vI
\\
&=
\left(
    \Sigma_{yy} 
    -
    \Sigma_{yy} \delta\trp (\Lambda + \delta\Sigma_{yy}\delta\trp)\inv
    \delta
    \Sigma_{yy} 
\right) 
\hadamard \vI
\label{eq:mstep:Rubin}
\\
&=
\left(
    \Sigma_{yy}\inv + \delta\trp\Lambda\delta
\right)\inv
\hadamard \vI
\end{align}
\eqref{eq:mstep:Rubin} is proposed by~\cite{Rubin1982}.

\bibliographystyle{plain}
\bibliography{emfa}
\end{document}
